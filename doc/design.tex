\documentclass{article}
\usepackage[utf8]{inputenc}
\usepackage[english]{babel}
\usepackage{fancyvrb}
\usepackage{hyperref}

\hypersetup{
    colorlinks=true,
    urlcolor=blue,
}

\title{\textbf{CS550 Advanced Operating Systems\\Programming Assignement 1\\Design Document}}
\author{Florentin \textsc{Bekier}\\Rémi \textsc{Blaise}}
\date{}

\begin{document}

\maketitle

\section{General architecture}

P2P File Sharing System is made of 2 softwares:

\begin{enumerate}
	\item The \textbf{Index} centralizes an index of all available files in the network. Each running instance of the Index creates a new independant file sharing network.
	\item The \textbf{Peer} can download and serve files from/to the network. Each running instance is a node of the network of the one Index it is connected to.
\end{enumerate}

\section{Index-Peer Communication Protocol}

The communication protocol between the Index and a Peer is built on top of the TCP layer.
They communicate in a \textbf{Peer-pull} only manner by sending JSON-formatted data through a socket in the following way:

\begin{enumerate}
	\item The Peer opens the connection over TCP.
	\item The Peer emits a request in the form of a Request JSON Message.
	\item The Index replies with a Response JSON Message.
	\item The Index closes the connection.
\end{enumerate}

\noindent The protocol is designed to enable the Index to expose an \textbf{API} made of \textbf{procedures}. \\
Here are the defined formats for the request and the response:

\subsection{Request JSON Message Format}

\begin{Verbatim}[tabsize=4]
{
	"name": "Name of the Procedure",
	"parameters": {
		…
	}
}
\end{Verbatim}

\subsection{Response JSON Message Format}

\begin{Verbatim}[tabsize=4]
{
	"status": "success|error",
	"data": {
		…
	} // if success
	"message": … // if error
}
\end{Verbatim}

\section{API of the Index}

Using the previously defined protocol, the Index provides the following API:

\subsection{The \texorpdfstring{\protect\Verb+registry+}{registry} procedure}

Enable a peer to register as a \textit{server} in the network, or to update its server information. \\
Information of a server is made of its UUID (choosen by the server, uniquely identify the server Peer in the network), its contact address (IP and port) and its file list. Files are identified in the network by their SHA-1 hash.

\vspace{1.0\baselineskip}

\noindent Expected request:

\begin{Verbatim}[tabsize=4]
{
	"name": "registry",
	"parameters": {
		"uuid": "string",
		"ip": "string",
		"port": "string",
		"files": []
	}
}
\end{Verbatim}

\noindent Expected response:

\begin{Verbatim}[tabsize=4]
{
	"status": "success",
	"data": null
}
\end{Verbatim}

\subsection{The \texorpdfstring{\protect\Verb+search+}{search} procedure}

Enable a peer to search for a file in the network. Return the list of peers delivering the file.

\noindent Expected request:

\begin{Verbatim}[tabsize=4]
{
    "name": "search",
    "parameters": {
        "fileId": "string"
    }
}
\end{Verbatim}

\noindent Expected response:

\begin{Verbatim}[tabsize=4]
{
    "status": "success",
    "data": [
        // List of the peer delivering the file
        {
            "uuid": "string",
            "ip": "string",
            "port": "string"
        },
        // ...
    ]
}
\end{Verbatim}

\section{Peer-Peer Communication Protocol}

A client-Peer and a server-Peer communicates by a protocol similar to the Index-Peer protocol. \\
The client requests of file and the server replies by sending back the requested file.

\begin{enumerate}
	\item The client opens the connection over TCP.
	\item The client emits the request in the form of a Request JSON Message.
	\item The server replies with a Response JSON Message.
	\item The server streams the file through the socket.
	\item The server closes the connection.
\end{enumerate}

\noindent Expected request:

\begin{Verbatim}[tabsize=4]
{
    "name": "retrieve",
    "parameters": {
        "fileId": "string"
    }
}
\end{Verbatim}

\noindent Expected response:

\begin{Verbatim}[tabsize=4]
{
    "status": "success",
    "data": {
        "filename": "string"
    }
}
\end{Verbatim}

\section{Client Command Line Interface}

The user of the Peer software has access to the following actions:

\begin{enumerate}
	\item List the current shared files served to the network.
	\item Download a file by hash. The file is downloaded into the shared folder and then available to other Peers in the network.
\end{enumerate}

\section{Technology Specification}

The 2 softwares are built using the following technologies:

\begin{itemize}
	\item The \href{https://developer.mozilla.org/en-US/docs/Web/JavaScript}{Javascript (ES6)} language interpreted by \href{https://nodejs.org/en/}{Node v12}. Given this project is network-oriented and has low performance requirements, this language is ideal for its very high level community-built libraries, making the development very straight-forward, as well as its asynchronous-oriented design, making the servers fully parallelized, able to handle multiple requests at the same time. Because both team members have a strong previous knowledge of this technology, this is an obvious choice.
	\item The \href{https://sqlite.org/index.html}{SQLite 3} database system handled by the \href{https://sequelize.org}{Sequelize} Object-Relational Mapper. This database system is selected for its simplicity of installation, the downside of this choice being not to have a querying interface as powerfull as traditional database system. Given the requirements of the project, the simplicity of the available SQL queries is not a problem.
\end{itemize}


\noindent Additional used tools:

\begin{itemize}
	\item \href{https://git-scm.com}{Git} version control manager hosted on \href{https://gitlab.com}{GitLab} platform.
	\item \href{https://www.npmjs.com}{npm} package manager.
\end{itemize}

\noindent The 2 softwares are made by following the usual Javascript coding style and software design conventions.

\section{Design of the Index}

The Index is built with the following software design:

\begin{itemize}
	\item The \Verb+app.js+ file is the entry point of the software and instanciates the TCP server.
	\item The \Verb+procedures.js+ file defines the actual procedures provided by the Index API.
	\item The \Verb+repository.js+ file handles the persistence layer by communicating with the database through the Sequelize ORM.
	\item The \Verb+config.js+ file reads the configuration.
\end{itemize}

\section{Design of the Peer}

The Peer software follows the given design:

\begin{itemize}
	\item The \Verb+app.js+ file is the entry point of the software and starts the Peer client and the Peer server.
	\item The \Verb+client.js+ file implements the client side: it registers the Peer to the Index, watches the shared folder and shows the CLI.
	\item The \Verb+server.js+ file implements the server side: it instanciates the TCP server and listens for file requests.
	\item The \Verb+interface.js+ file implements the communication protocol.
	\item The \Verb+files.js+ file handles the file sharing management.
	\item The \Verb+config.js+ file reads and initializes the configuration.
\end{itemize}

\noindent Both the Index and the Peer are designed to be easy to install and to configure.

\section{Security}

From a security point of vue, the system could be vulnerable to the following attacks:

\begin{enumerate}
	\item A maliciours tier sends faulty request data to make the server-Peer or the Index crash.
	\item A man-in-the-middle corrupts the file shared by the server-Peer to the client-Peer.
	\item A malicious tier registers erroneous server-Peer information ot the Index.
\end{enumerate}

\noindent In order to address these vulnerabilities, we carefully design the following defenses:

\begin{enumerate}
	\item Validate any data received from a client so it doesn't make the server crash.
	\item Identify the files by their SHA-1 hash so the client-Peer can verify the file is not corrupted.
	\item Provide a public key at the first registration and cryptographically sign any further call to the \Verb+register+ procedure.
\end{enumerate}

\section{Improvements to the softwares}

To go further, the following improvements could be brought to our P2P system:

\begin{itemize}
	\item The Index, as is, doesn't support scaling given that Index data is centralized in a single database. An improvement could be to make it able to be installed on a distributed system.
	\item A GUI could replace the current CLI of the Peer, making it usable by common users.
	\item The downloading of big files could use different server-Peers if available. Similarly, the index could sort available server-Peers in order to better distribute the global charge over all servers.
	\item The Index could include a web-browsable interface listing available files on the system.
	\item The Index could reduce the risk of trying to contact an unavailable file server by periodically checking the availability of hosts or removing servers from the list after a given timeout.
\end{itemize}
	
\end{document}